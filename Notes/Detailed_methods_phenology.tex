\documentclass{article}\usepackage[]{graphicx}\usepackage[]{color}

\usepackage{alltt}
\usepackage{float}
\usepackage{graphicx}
\usepackage{tabularx}
\usepackage{siunitx}
\usepackage{amssymb} % for math symbols
\usepackage{amsmath} % for aligning equations
\usepackage{textcomp}
\usepackage{natbib}
\topmargin -1.5cm        
\oddsidemargin -0.04cm   
\evensidemargin -0.04cm
\textwidth 16.59cm
\textheight 21.94cm 
%\pagestyle{empty} %comment if want page numbers
\parskip 7.2pt
\renewcommand{\baselinestretch}{1.5}
\parindent 0pt
\usepackage{blindtext}
\usepackage[T1]{fontenc}
\usepackage[utf8]{inputenc}

\title{Detailed phenology methods }

\author{Deirdre Loughnan}

\begin{document}

\maketitle

Through personal communications with local land managers and foresters across our research sites, dominant angiosperm woody plant species native to deciduous forests in British Columbia were identified for potential use in this study. I collected samples from two populations in British Columbia: Manning Provincial Park (40.1150\textdegree{} N, 120.8558 \textdegree{} W, and parks surrounding the town of Smithers (54.7824\textdegree{} N, 127.1686\textdegree{} W). During the 2019 growing season, I located and tagged individuals of 21 of the listed species within each of these populations. Several species were excluded from this study either due to concerns over accurate identification in the field, as was the case for \emph{Salix} species, or safety concerns over the spiny nature of \emph{Oplopanax horridus}. When possible, I collected samples from individuals for both this phenological experiment and collection of functional traits, but additional individuals were identified using winter identification guides to reach the desired sample size of eight individuals per treatment. I collected branch cuttings 40 cm in length during the last two weeks of October 2019 across multiple forest stands near Smithers and in Manning Provincial Park, British Columbia. Of the species, \emph{Betula papyrifera} was only present in Smithers, while \emph{Rhododendron albiflorum} was only present in Manning Park. Due to forest management practices and road maintenance, I was also unable to collect samples of \emph{Sambucus racemosa} from Smithers. Samples from tree species were only collected from healthy individuals that were accessible using a 5m pole pruner from the ground and stored on ice during collection and transport. I collected multiple branches from individuals when possible, with individual identity recorded. Samples were collected from the Smithers area from October 20 to October 23rd, 2019. On October 24th the samples were driven back to the UBC campus, where they were immediately placed in the chilling chamber set to 4\textdegree{}C. Samples were collected from Manning Provincial Park from October 26 to October 28th, when they too were placed in the chilling chamber. 

Using a fully factorial design, cuttings were randomly assigned to eight experimental treatments consisting of: two chilling durations of either three or seven weeks at 4\textdegree{}C, two forcing conditions representing either warm spring conditions (20\textdegree{}C:10\textdegree{}C) or cool spring conditions (15\textdegree{}C:5\textdegree{}C), and two photoperiods of either 8 or 12 hours of daylength. For my chilling treatment I used a Conviron PGV-35, while eight chambers were used for the forcing and photoperiod treatments, five of which were Conviron E-15 and three Conviron PGR-15. A combination of red-far red LED and fluorescent lights were used in each chamber, with the chamber floor level adjusted to achieve a light intensity of approximately 400 \mu mol. The experiment was setup from October 30th to November 2nd. Cuttings were placed in 500ml Erlenmeyer flasks with 400 ml of water, with four randomly assigned individuals assigned to each flask based on treatment groups. I changed the water and re-cut each sample every seven days to prevent bacterial growth and reduce the effects of callusing on cutting health. To reduce any effects of individual chambers or flask position, all samples were chilled within one chamber and I randomly rotated the chambers used for forcing and chilling treatments every two weeks, with flask position within all chambers rotated weekly. To assess the phenology of cuttings, I used the BBCH (Biologische Bundesanstalk, Bundessortenamt and CHemical Industry) scale modified for my focal species \citep{Finn2007}. The percent of buds at the three highest stages of budburst were visually estimated for each day of observation and the estimated final total percent budburst recorded. Observations were performed every 2-3 days for the first 12 weeks of the experiment, after which the low chill samples were moved out of the chambers and into a greenhouse for an additional four weeks. During this final period, observations were made every four days. The high chill samples were only observed in the growth chambers for 12 weeks, due to restrictions put on research in spring 2020, during the covid-19 pandemic. Day of budburst was considered the day on which 80\% of buds reached stage seven of the BBCH scale for most species, however, I adjusted for species with considerably lower percentages by using lower thresholds. I measured chamber temperatures for the majority of the experiment using HOBO pendant data loggers. The temperature in the chilling chamber was on average X \textdegree{}C, while the average temperature across the eight treatment chambers ranged from X-X \textdegree{}C. On day 56 of the experiment one chamber malfunctioned and was shut off for a four hour period, during which the temperature rose to X\textdegree{}C. 

Chill portions were calculated using the chillR package (citation). For our western sites, we used data from Environment Canada, from September 1, 2019 to the day of collection for each site (October 24, 2019 and October 29, 2019 for our Smithers and Manning Park sites respectively. Climate data for our eastern sites was obtained from XX and data from September 1, 2014 to the dates of collection


\end{document}