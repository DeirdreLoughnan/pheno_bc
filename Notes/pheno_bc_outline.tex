\documentclass[11pt,letter]{article}
\usepackage[top=1.00in, bottom=1.0in, left=1.1in, right=1.1in]{geometry}
\renewcommand{\baselinestretch}{1.1}
\usepackage{graphicx}
\usepackage{natbib}
\usepackage{amsmath}

\def\labelitemi{--}
\parindent=0pt


\begin{document}
%\bibliographystyle{/Users/Lizzie/Documents/EndnoteRelated/Bibtex/styles/besjournals}
\renewcommand{\refname}{\CHead{}}

{\bf Forcing and chilling equally important in spring phenology in western forest communities}\\ 

\section{Introduction} 

Plant species are experiencing novel selective pressures in light of climate change, with one of the most apparent responses being their change in phenologies. 
\begin{itemize}
\item Studies have observed advances in the timing of species phenologies (Menzel et al. 2006) 
\item Changes seem to be greater in the spring than the fall
\item It is important to understand and be able to predict these changes, as advances in spring phenology impacts community structures, the length of the growing season, and ultimately global climate cycles (Richardson et al. 2009).

\item Not all species are advancing some are experiencing delays -- while this variation can reflects the response to varying selective pressures, biotic interaction, recent research has largely focused on the abiotic drivers of plant phenologies and interspecific variation in cue requirements. 
\item In the case of woody plant species, three cues have been shown to drive the timing of budburst and leafout in temperate systems. 
	\begin{enumerate}
	\item Chilling
	\item Forcing
	\item Photoperiod
	\item Species vary in their relative sensitivity to different cues
	\item Photo period is argued by some to not be a key driving cue, but there is evidence of species, F. sylvatica for which it is important
	\item Cues interact and can compensate for one another: insufficient chilling can be overcome by high forcing and long photoperiods
	\end{enumerate}

Differences between shrubs and trees:
\begin{enumerate}
\item Donnelly's work: shrubs in the understory are earlier
\end{enumerate}

Differences between the timing of lateral and terminal budburst:
\end{itemize}


\section{Research Objectives}
To my knowledge, this study will be the first community wide study of woody plant phenology in British Columbia. In performing this larger scale growth chamber study, I will answer the following research questions:
\begin{enumerate}
\item Do species sensitivities to chilling, forcing, and photoperiod differ between species and how?
\item Are the phenologies of some species driven by a single cue or do some require a combination of cues?
\item How do cues differ in species across North America? Do forest communities in British Columbia differ in their cues compared to central North America? 
\item What are the differences between functional groups - trees vs shrubs?
\item Do we see differences in congener species? Do we see phylogenetic effects? 
\end{enumerate}

\section{The model}

\begin{align*}
\hat{y}_i &=  \mu_{grand} + \alpha_{sp_i} + \beta F_{sp_i} \times F_i + \\
& \beta C_{sp_i}  \times C_i + \beta P_{sp_i}  \times P_i + \\
& \beta d2 \times d2 + \beta d3 \times d3 + \beta d4 \times d4 +\\
& \beta FP_{sp_i}  \times FP_i  + \beta FC_{sp_i}  \times FC_i + \beta CP_{sp_i}  \times CP_i + \\
& \beta FT _{sp_i} + \beta CT _{sp_i} + \beta PT _{sp_i}\\
& \beta Fd2 \times Fd2 + \beta Fd3 \times Fd3 + \beta Fd4 \times Fd4 +\\
& \beta Cd2 \times Cd2 + \beta Cd3 \times Cd3 + \beta Cd4 \times Cd4 +\\
& \beta Pd2 \times Pd2 + \beta Pd3 \times Pd3 + \beta Pd4 \times Pd4 \\
\vspace{1ex}\\
\beta FP_{sp} &= \alpha FP + \beta FPraw_{sp}  \times \sigma FP\\
\vspace{1ex}\\
\beta_{F} & \sim N(\mu_{site}, \sigma_{F}) \\
\beta_{C} & \sim N(\mu_{site}, \sigma_{C}) \\
\beta_{P} & \sim N(\mu_{site}, \sigma_{P}) \\
....\\
y & \sim N(\hat{y}, \sigma_{y}) \\
\end{align*}

\begin{align*}
\hat{y}_i &=  \mu_{grand} + \alpha_{sp_i} + \beta F_{sp_i} \times F_i + \\
& \beta C_{sp_i}  \times C_i + \beta P_{sp_i}  \times P_i + \beta Transect  + \\
& \beta FP_{sp_i}  \times FP_i  + \beta FC_{sp_i}  \times FC_i + \beta CP_{sp_i}  \times CP_i + \\
& \beta FT + \beta CT  + \beta PT  \\
\vspace{1ex}\\
\beta FP_{sp} &= \alpha FP + \beta FPraw_{sp}  \times \sigma FP\\
\vspace{1ex}\\
\beta_{F} & \sim N(\mu_{site}, \sigma_{F}) \\
\beta_{C} & \sim N(\mu_{site}, \sigma_{C}) \\
\beta_{P} & \sim N(\mu_{site}, \sigma_{P}) \\
....\\
y & \sim N(\hat{y}, \sigma_{y}) \\
\end{align*}


\section{Results}

Key results to include:
How many cuttings did not break bud -- table of proportions of success
What happens when you remove species with the lowest success?
Can we conclude that factors offset each other if the interaction becomes negative?

\section{Figures \& Tables}

\begin{itemize}
\item Figure of model estimates
\item Figure with model estimates and species-level effects with 50\% CI
\item Table of mean bb dates for each species by site
\item Table of model output (mean, sd, 2.5, 50, 97.5, Rhat! 
\item Table of \% bb and leafout for each species
\end{itemize}

\section{To-do:}
\begin{itemize}
\item chill units - can I get the data to calculate chill units in the field, calculate for the growth chamber
\item Phylogeny - ask Simon Joli if he will make a phylogeny for me!
\end{itemize}

\end{document}