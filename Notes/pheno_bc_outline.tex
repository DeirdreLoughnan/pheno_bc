\documentclass[11pt,letter]{article}
\usepackage[top=1.00in, bottom=1.0in, left=1.1in, right=1.1in]{geometry}
\renewcommand{\baselinestretch}{1.1}
\usepackage{graphicx}
\usepackage{natbib}
\usepackage{amsmath}

\def\labelitemi{--}
\parindent=0pt


\begin{document}
%\bibliographystyle{/Users/Lizzie/Documents/EndnoteRelated/Bibtex/styles/besjournals}
\renewcommand{\refname}{\CHead{}}

{\bf Forcing and chilling equally important in spring phenology in western forest communities}\\ 

\section{Research Objectives}
To my knowledge, this study will be the first community wide study of woody plant phenology in British Columbia. In performing this larger scale growth chamber study, I will answer the following research questions:
\begin{enumerate}
\item Do species sensitivities to chilling, forcing, and photoperiod differ between species and how?
\item Are the phenologies of some species driven by a single cue or do some require a combination of cues?
\item How do cues differ in species across North America? 
\item Do we see differences in congener species? 
\end{enumerate}

\section{Results}

Key results to include:
How many cuttings did not break bud -- table of proportions of success
What happens when you remove species with the lowest success?

Can conclude that factors offset each other if the interaction becomes negative



\section{Figures \& Tables}

\begin{itemize}
\item Figure of model estimates
\item Figure with model estimates and species-level effects with 50\% CI
\item Table of mean bb dates for each species by site
\item Table of model output (mean, sd, 2.5, 50, 97.5, Rhat! 
\item Table of \% bb and leafout for each species
\end{itemize}

\section{To-do:}
\begin{itemize}
\item chill units - can I get the data to calculate chill units in the field, calculate for the growth chamber
\item Phylogeny - ask Simon Joli if he will make a phylogeny for me!
\end{itemize}

\end{document}