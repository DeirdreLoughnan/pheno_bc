\documentclass[11pt,a4paper]{article}
\usepackage[top=1.00in, bottom=1.0in, left=1.1in, right=1.1in]{geometry}
\usepackage{graphicx}
\usepackage[sort&compress, super, numbers]{natbib}
\usepackage[export]{adjustbox}

% novelty statement, a cover letter, and a title page including an authorship statement, a data statement, the number of words in abstract/main text and the number of figure/table/boxes.
% recommended and opposed reviewers, and recommended and opposed editorial board members with whom they are in conflict. 
\begin{document}

\pagenumbering{gobble}

\noindent \includegraphics[width=0.4\textwidth, right]{letterhead/Facultyofforestry.png}
\noindent Dear Dr. Thrall:
\vspace{1.5ex}\\
\noindent Please consider our paper, ``Current environments and evolutionary history shape forest temporal assembly'' for publication as a letter in \emph{Ecology Letters}. 
\vspace{1.5ex}\\ 
%emw26Nov: the below paragraph save for the last sentence reads too much like the abstract -- cover letters need to be unique from abstracts, so this does not really work. 
\noindent Climate change is impacting species phenologies---timing of life history events---reshaping communities, and altering ecosystem functioning. While most species are advancing phenologically, there is considerable species-level variation in the rate of these changes. Theory suggests this variation comes from species partitioning time to reduce competition for resources. But shared evolutionary history and differences in environmental cues may further moderate the temporal assembly of a community. Understanding this variability is critical to accurate forecasts of future community dynamics and requires large-scale community-level experiments to identify the drivers of community assembly. %emw26Nov: I like the idea of this last sentence, but I would tweak it more to establish the challenge -- getting at this requires experiments of environmental cues, which are hard and so are traditionally done for only a few species, making a community perspective impossible ... so far ... then below you can say you overcome this and present amazing new results!
\vspace{1.5ex}\\
%emw26Nov: I would tuck evolutionary history into the meat of the paragraph not the leading sentence... 
\noindent Focusing on woody plants for which phenological cues---specifically temperature and photoperiod---are well known, we directly test the role of populations, species and community-level variation, as well as evolutionary history in shaping phenologies. We conducted two large-scale controlled environment studies in which we observed leafout in 47 tree and shrub species from four populations under variable temperature and photoperiod cues. 
%emw26Nov: Below is a great point, but I might try to write two short paragraphs about the biology findings first THEN have a quick paragraph saying that you do this in an important and cool way that applies to many systems.
In pairing the results of this experimental data with a Bayesian phylogenetic model, we are one of the first to mechanistically test the relationships and variability in leafout cues and evolutionary relationships, using a powerful analytical approach that has broad applications across diverse species assemblages and phenologies.
\vspace{1.5ex}\\
%emw26Nov: For ELE we need to be less specific to phenology and forests ... see my edits to the abstract and try to edit the below to be more general. You might review some of Volker Rudolf's papers on temporal assembly for inspiration. I would focus broader and then tuck in a sentence about forest assembly and climate change versus leading with leafout phenology in any paragraph. 
\noindent  Our findings provide novel insight into the drivers of the high species-level variation we observe in leafout phenology. We observed very little site-level variation in leafout timing, but considerable species-level variation, as cues generally led to advances in budburst.  While many models critical for future forecasts assume temperature and light explain forest leafout, we found cues to only explain between 38.4 and 67.6\% of variation. The remaining variation, which is partially explained by evolutionary history, suggests our understanding of leafout phenology is incomplete and that other unidentified traits or cues also shaping the temporal assembly of forest communities. %emw26Nov: Again, it can really annoy editors when you copy too much of the abstract. Remember -- they will read the abstract and your cover letter. Be sure to maximize what you given them in these two tiny opportunities to connect with an editor, we should not repeat ourselves much. 
\vspace{1.5ex}\\ 
\noindent All authors contributed to this work and approved this version for submission. The manuscript is 3773 words with a 150 word summary, and 3 figures. It is not under consideration elsewhere. We hope you find it suitable for publication in \emph{Ecology Letters}, and look forward to hearing from you. 
\vspace{1.5ex}\\
\noindent We recommend the following reviewers: Dr. Meredith Zettlemoyer, Dr. Mason Heberling, Dr. Jason Fridley, Dr. Rong Yu, and Dr. Ameila Caffara. 
\vspace{1.5ex}\\
\noindent Sincerely, \\
\includegraphics[scale=.2]{letterhead/shot.png} \\ %emw26Nov: What about making your signature a little less tiny? You could also add that you're a postdoc 
\noindent Deirdre Loughnan\\
\noindent Postdoctoral Fellow, Department of Zoology\\
\noindent University of British Columbia

\newpage

\vspace{-5ex}
\bibliographystyle{refs/bibstyles/nature.bst}% 

\bibliography{refs/traitors_mar23.bib}
\newpage


\end{document}