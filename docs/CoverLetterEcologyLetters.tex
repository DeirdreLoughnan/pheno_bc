\documentclass[11pt,a4paper]{article}
\usepackage[top=1.00in, bottom=1.0in, left=1.1in, right=1.1in]{geometry}
\usepackage{graphicx}
\usepackage[sort&compress, super, numbers]{natbib}
\usepackage[export]{adjustbox}

% novelty statement, a cover letter, and a title page including an authorship statement, a data statement, the number of words in abstract/main text and the number of figure/table/boxes.
% recommended and opposed reviewers, and recommended and opposed editorial board members with whom they are in conflict. 
\begin{document}

\pagenumbering{gobble}

\noindent \includegraphics[width=0.4\textwidth, right]{letterhead/Facultyofforestry.png}
\noindent Dear Dr. Thrall:
\vspace{1.5ex}\\
\noindent Please consider our paper, ``Current environments and evolutionary history shape forest temporal assembly'' for publication as a letter in \emph{Ecology Letters}. 
\vspace{1.5ex}\\ 
Climate change has altered the cues shaping ecological communities, leading to novel species interactions and community structure. The strongest evidence of these changes is from species' temporal niche---particularly the timing of their life history events or phenologies---which in general have advanced in response to warming. Previous studies have found phenology to be highly variable across species and interannually, but most species' phenological cues are still poorly understood. To mechanistically understand the causes of phenological shifts and variability, we can use experiments in which we vary environmental cues, but this type of study is logistically challenging and to date, are conducted on only a few species. This lack of large-scale, community-level experiments, limits our ability to accurately forecast the composition and dynamics of communities under future climates.
\vspace{1.5ex}\\
%emw26Nov: I would tuck evolutionary history into the meat of the paragraph not the leading sentence... 
\noindent 
Our study is one of the first to directly test the role of populations, species and community-level variation in shaping phenologies. Drawing from decades of previous research, we used woody plants for which phenological cues---specifically temperature and photoperiod---are well studied. We conducted two large-scale controlled environment studies in which we observed leafout in 47 tree and shrub species from four populations under variable temperature and photoperiod cues.  To further account for the effects of species evolutionary history, we paired our experimental data with a phylogeny of our species and used a Bayesian phylogenetic model to mechanistically test the relationships and variability in phenological cues and evolutionary relationships.
\vspace{1.5ex}\\
\noindent  
In general, we found greater cues to lead to earlier phenologies, but with very little site-level variation. The considerable phenological differences observed across species were partially explained by evolutionary history, while the three cues thought to best explain leafout in woody plant only explained between 38.4 and 67.6\% of variation. This suggests our understanding of leafout---a well studied phenological event---is still incomplete and that other unidentified traits or cues also shaping the temporal assembly of communities. 
\vspace{1.5ex}\\
In addition to providing novel insights of how cues and species evolutionary history shape species temporal niche, our study presents a powerful analytical approach that has broad applications across diverse species assemblages and phenologies.
\vspace{1.5ex}\\
%emw26Nov: For ELE we need to be less specific to phenology and forests ... see my edits to the abstract and try to edit the below to be more general. You might review some of Volker Rudolf's papers on temporal assembly for inspiration. I would focus broader and then tuck in a sentence about forest assembly and climate change versus leading with leafout phenology in any paragraph. 
\noindent All authors contributed to this work and approved this version for submission. The manuscript is 3773 words with a 150 word summary, and 3 figures. It is not under consideration elsewhere. We hope you find it suitable for publication in \emph{Ecology Letters}, and look forward to hearing from you. 
%\noindent We recommend the following reviewers: Dr. Meredith Zettlemoyer, Dr. Mason Heberling, Dr. Jason Fridley, Dr. Rong Yu, and Dr. Ameila Caffara. 
\vspace{1.5ex}\\
\noindent Sincerely, \\
\includegraphics[scale=.4]{letterhead/shot.png} \\ 
\noindent Deirdre Loughnan\\
\noindent Sentinels of Change Postdoctoral Fellow\\
\noindent Hakai Institute $|$ Department of Zoology\\
\noindent University of British Columbia
\newpage
\vspace{-5ex}
\bibliographystyle{refs/bibstyles/nature.bst}% 

\bibliography{refs/traitors_mar23.bib}
\newpage


\end{document}