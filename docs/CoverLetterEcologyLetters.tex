\documentclass[11pt,a4paper]{article}
\usepackage[top=1.00in, bottom=1.0in, left=1.1in, right=1.1in]{geometry}
\usepackage{graphicx}
\usepackage[sort&compress, super, numbers]{natbib}
\usepackage[export]{adjustbox}

% novelty statement, a cover letter, and a title page including an authorship statement, a data statement, the number of words in abstract/main text and the number of figure/table/boxes.
% recommended and opposed reviewers, and recommended and opposed editorial board members with whom they are in conflict. 
\begin{document}

\pagenumbering{gobble}

\noindent \includegraphics[width=0.4\textwidth, right]{letterhead/Facultyofforestry.png}
\noindent Dear Dr. Thrall:
\vspace{1.5ex}\\
\noindent Please consider our paper, ``Current environments and evolutionary history shape forest temporal assembly'' for publication as a letter in \emph{Ecology Letters}. 
\vspace{1.5ex}\\ 
\noindent Climate change is impacting species phenologies---timing of life history events---reshaping communities, and altering ecosystem functioning. While most species are advancing phenologically, there is considerable species-level variation in the rate of these changes. Theory suggests this variation comes from species partitioning time to reduce competition for resources. But shared evolutionary history may further moderate such potential effects. Understanding this variability is critical to accurate forecasts of future forest dynamics and sequestration, and requires large-scale community-level experiments to identify the drivers of forest temporal assembly.
\vspace{1.5ex}\\
\noindent While many models critical for future forecasts assume temperature and light explain forest leafout, we directly test the role of populations, species and community-level variation, including the role of evolutionary history in shaping species differences. Focusing on woody plants---for which temperature and photoperiod cues are well known---we tested for differences in the start of leafout at the community-level—with 47 tree and shrub species—and between four populations that span North America. Pairing the results of these controlled environment studies with a Bayesian phylogenetic model, we are one of the first to mechanistically test the relationships and variability in leafout cues and evolutionary relationships. This work also showcases an analytical approach that has broad applications across diverse species assemblages and phenologies.
\vspace{1.5ex}\\
\noindent  Our findings provide novel insight into the drivers of the high species-level variation we observe in leafout phenology. We observed greater cues to lead to advances in budburst, with some effects of evolutionary history, but little variation across our four sites. This suggests our understanding of leafout phenology is incomplete and that other unidentified traits or cues also shaping the temporal assembly of forest communities. 
\vspace{1.5ex}\\ 
\noindent All authors contributed to this work and approved this version for submission. The manuscript is 3773 words with a 150 word summary, and 3 figures. It is not under consideration elsewhere. We hope you find it suitable for publication in \emph{Ecology Letters}, and look forward to hearing from you. 
\vspace{1.5ex}\\
\noindent We recommend the following reviewers: Dr. Jason Fridley, Dr. Meredith Zettlemoyer, Dr. Mason Heberling, Dr. Rong Yu, and Dr. Ameila Caffara. 

\vspace{1.5ex}\\
\noindent Sincerely, \\
\includegraphics[scale=.2]{letterhead/sigDL.png} \\
\noindent Deirdre Loughnan\\
\noindent Zoology\\
\noindent University of British Columbia

\newpage

\vspace{-5ex}
\bibliographystyle{refs/bibstyles/nature.bst}% 

\bibliography{refs/traitors_mar23.bib}
\newpage


\end{document}