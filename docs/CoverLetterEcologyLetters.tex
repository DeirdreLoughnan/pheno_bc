\documentclass[11pt,a4paper]{article}
\usepackage[top=1.00in, bottom=1.0in, left=1.1in, right=1.1in]{geometry}
\usepackage{graphicx}
\usepackage[sort&compress, super, numbers]{natbib}
\usepackage[export]{adjustbox}

% novelty statement, a cover letter, and a title page including an authorship statement, a data statement, the number of words in abstract/main text and the number of figure/table/boxes.
% recommended and opposed reviewers, and recommended and opposed editorial board members with whom they are in conflict. 
\begin{document}

\pagenumbering{gobble}

\noindent \includegraphics[width=0.4\textwidth, right]{letterhead/Facultyofforestry.png}
\noindent Dear Dr. Thrall:
\vspace{1.5ex}\\
\noindent Please consider our paper, ``Current environments and evolutionary history shape forest temporal assembly'' for publication as a letter in \emph{Ecology Letters. 
\vspace{1.5ex}\\ 
%emw2024Aug8 -- I'd skip proximate and ultimate in the cover letter; it's too confusing for such a short space. And go big or go home so I edited for that. 

\noindent Climate change is impacting species phenologies---timing of life history events---reshaping communities, and altering ecosystem functioning \citep{Cleland2007a,Beard2019,Gu2022}. Increasing research suggests these changes are linked, as species that shift the most phenologically with warming appear to alter the assembly of communities and out-compete later species, either through their phenologies or growth strategies. Understanding this is critical to accurate forecasts, but extremely challenging because of high variability in observational phenology. The result is that plant phenology has been systematically omitted from global trait frameworks that link to plant growth strategies and can help predict future community dynamics and ecosystem functioning.
\vspace{1.5ex}\\
\noindent Here we overcome these challenges by combining global data from experiments on budburst phenology and plant traits, with cutting-edge Bayesian approaches that can jointly model budburst timing in response to environmental cues and in relation to major plant traits. Our dataset represents one of the most comprehensive datasets of trait syndrome available, making it an important first step to identify general trends that scale across populations and species. Further, by using a joint modelling approach, we are the first to identify broader trait relationships to phenological cues based on species-level trait variation, while also accounting for the high degree of uncertainty that arises when combining datasets of diverse communities and locations.
\vspace{1.5ex}\\
\noindent  Our findings show how traits and phenologies are inextricably linked to strategies for growth. Earlier species exhibit acquisitive traits---shorter maximum heights, and denser, lower nitrogen leaves---while later-active species are taller with low nitrogen leaves. Our results fit budburst phenology firmly within major functional trait frameworks to allow us to tease apart the underlying mechanisms shaping species phenology and traits across communities, and provide novel insights that can be used to better predict how communities may shift in their growth strategies alongside changing phenology with climate change. 
\vspace{1.5ex}\\ %5000 word max 6 figures
\noindent All authors contributed to this work and approved this version for submission. The manuscript is XXX words with a ZZZ word summary, and X figures. It is not under consideration elsewhere. We hope you find it suitable for publication in \emph{Ecology Letters}, and look forward to hearing from you. 
\vspace{1.5ex}\\
\noindent We recommend the following reviewers: Dr. Jason Fridley, Dr. Meredith Zettlemoyer, Dr. Mason Heberling, Dr. Rong Yu, and Dr. Ameila Caffara. 

\vspace{1.5ex}\\
\noindent Sincerely, \\
\includegraphics[scale=.2]{letterhead/sigDL.png} \\
\noindent Deirdre Loughnan\\
\noindent Zoology\\
\noindent University of British Columbia

\newpage

\vspace{-5ex}
\bibliographystyle{refs/bibstyles/nature.bst}% 

\bibliography{refs/traitors_mar23.bib}
\newpage


\end{document}