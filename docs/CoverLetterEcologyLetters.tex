\documentclass[11pt,a4paper]{article}
\usepackage[top=1.0in, bottom=1.0in, left=1.1in, right=1.1in]{geometry}
\usepackage{graphicx}
\usepackage[sort&compress, super, numbers]{natbib}
\usepackage[export]{adjustbox}

% novelty statement, a cover letter, and a title page including an authorship statement, a data statement, the number of words in abstract/main text and the number of figure/table/boxes.
% recommended and opposed reviewers, and recommended and opposed editorial board members with whom they are in conflict. 
\begin{document}

\pagenumbering{gobble}

\noindent \includegraphics[width=0.4\textwidth, right]{letterhead/Facultyofforestry.png}
\noindent Dear Dr. Thrall:
\vspace{1.5ex}\\
\noindent Please consider our paper, ``Current environments and evolutionary history shape forest temporal assembly'' for publication as a letter in \emph{Ecology Letters}. 
\vspace{1.5ex}\\ 
% Ecological communities are shaped by environmental variability and species interactions across both time and space. 
Global shifts in the timings of species life history events with climate change has led to increasing interest in how communities assemble in time. Yet progress has been slow, as species timings exhibit variation across space and time, with especially high inter-annual variability in systems where shifts have been greatest. Using experiments we can decompose this variability into predictable responses to environmental cues, like temperature and daylength. But given their logistic challenges, most experiments have focused on only a few species, and provide limited insights into community dynamics under future climates. % make shorter add 4-6 high impact citations
\vspace{1.5ex}\\
\noindent 
Here, we present a large-scale experiment, spanning four sites, 47 species and over 3450 individual cuttings, to test the cues of temporal assembly.  We focus on leafout of trees and shrubs for which the environmental cues---temperature and photoperiod---are well studied and directly test the variability at the population, species and community-level. To mechanistically test the relationships and variability in environmental cues and species evolutionary relationships, we paired our experimental data with a phylogenetic tree and Bayesian model. 
\vspace{1.5 ex}\\
\noindent  We found species differed their timings in each site, supporting the idea that temporal assembly may structure communities. Further, we show that species timings can be robustly decomposed into stable responses to temperature and daylength, with responses surprisingly constant across populations---even those separated by 6\textdegree  of latitude and 55\textdegree  of longitude. But responses to cues explained little of the temporal variation across species. Differences in species timings were also explained by intrinsic differences, which themselves were related to evolutionary history. These findings suggest our understanding of one of the most well studied events---leafout---is still incomplete and that other unidentified traits or cues shape communities temporal assembly. 
\vspace{1.5ex}\\
Our study also presents a powerful analytical approach that has broad applications across diverse species assemblages. The phylogenetic model we could be applied to other types of life history events or suites of species, allowing us to test for differences between invasive and native species in a community or across species in distinct trophic-levels or functional groups.
\vspace{1.5ex}\\
%emw26Nov: For ELE we need to be less specific to phenology and forests ... see my edits to the abstract and try to edit the below to be more general. You might review some of Volker Rudolf's papers on temporal assembly for inspiration. I would focus broader and then tuck in a sentence about forest assembly and climate change versus leading with leafout phenology in any paragraph. 
\noindent Both authors contributed to this work and approve this version for submission. The manuscript is 3773 words with a 150 word summary, and 3 figures and is not under consideration elsewhere. We hope you find it suitable for publication in \emph{Ecology Letters}, and look forward to hearing from you. 
%\noindent We recommend the following reviewers: Dr. Meredith Zettlemoyer, Dr. Mason Heberling, Dr. Jason Fridley, Dr. Rong Yu, and Dr. Ameila Caffara. 
\vspace{1.5ex}\\
\noindent Sincerely, \\
\includegraphics[scale=.4]{letterhead/shot.png} \\ 
\noindent Deirdre Loughnan\\
\noindent Sentinels of Change Postdoctoral Fellow\\ %emw19Dec: Nice!
\noindent Hakai Institute $|$ Department of Zoology\\
\noindent University of British Columbia
\newpage
\vspace{-5ex}
\bibliographystyle{refs/bibstyles/nature.bst}% 

\bibliography{refs/traitors_mar23.bib}
\newpage


\end{document}