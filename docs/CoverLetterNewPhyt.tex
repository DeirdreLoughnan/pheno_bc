\documentclass[11pt,a4paper]{article}
\usepackage[top=1.0in, bottom=1.0in, left=1.1in, right=1.1in]{geometry}
\usepackage{graphicx}
\usepackage[sort&compress]{natbib}
\usepackage[export]{adjustbox}
\usepackage[ngerman]{babel}
\usepackage[utf8]{inputenc}
\usepackage[T1]{fontenc}
\renewcommand{\bibname}{References}%names reference list 
% novelty statement, a cover letter, and a title page including an authorship statement, a data statement, the number of words in abstract/main text and the number of figure/table/boxes.
% recommended and opposed reviewers, and recommended and opposed editorial board members with whom they are in conflict. 
\begin{document}

\pagenumbering{gobble}

\noindent \includegraphics[width=0.4\textwidth, right]{letterhead/ubc-logo-2018-fullsig-blue-cmyk.png}
\noindent Dear Dr. Öpik
\vspace{1.5ex}\\
\noindent Please consider our paper, ``How temperature, photoperiod and evolutionary history shape forest leafout'' for publication as a full paper in \emph{New Phytologist}. 
\vspace{1.5ex}\\ 
Global shifts in the timings of species life history events with climate change has led to increasing interest in how communities assemble in time \citep{Cope2022, Cleland2024}. Yet progress has been slow, as species timings are highly variable, especially in systems where shifts have been greatest. Using experiments we can decompose this variability into predictable responses to environmental cues, such as temperature and daylength \citep{Basler2014,Vitasse2014,Zohner2016}. Given their logistical challenges, however, most experiments have focused on only a few species---providing limited insights into community dynamics under future climates.
\vspace{1.5ex}\\ 
\noindent \textit{What hypotheses or questions does this work address?} We tested how environmental cues and species differences shape the temporal assembly of woody plants across North America. Focusing on leafout of trees and shrubs---for which environmental cues are known---we estimate the variability at the population, species and community-levels and mechanistically test the relationships and variability in environmental cues and species evolutionary relationships. %53 words
\vspace{1.5ex}\\ 
\noindent \textit{How does this work advance our current understanding of plant science?}  By using a large-scale experiment of 47 species from communities spanning 6\textdegree latitude and 55\textdegree longitude, our results provide the best evidence for the similarities in leafout across temperate forests and functional groups. But species-level responses to cues, and moderate phylogenetic structuring, only explained 38-67\% of variation, suggesting our understanding of leafout timing is incomplete. %57 words
\vspace{1.5ex}\\
\noindent \textit{Why is this work important and timely?} We found systematic differences in species' timings in each site, supporting the idea that temporal assembly may structure communities, along with species intrinsic differences and evolutionary history. Our study also presents a powerful analytical approach that can be broadly applied to other events or suites of species, including species invasions or trophic-levels. %52 words
\vspace{1.5ex}\\
\noindent Both authors contributed to this work and approve this version for submission. The manuscript is 4095 words with a 199 word summary, and 4 figures and is not under consideration elsewhere. We hope you find it suitable for publication in \emph{New Phytologist}, and look forward to hearing from you. 
%\noindent We recommend the following reviewers: Dr. Meredith Zettlemoyer, Dr. Mason Heberling, Dr. Jason Fridley, Dr. Rong Yu, and Dr. Ameila Caffara. 
\vspace{1.5ex}\\
\noindent Sincerely, \\
\includegraphics[scale=.4]{letterhead/shot.png} \\ 
\noindent Deirdre Loughnan\\
\noindent Sentinels of Change Postdoctoral Fellow\\ %emw19Dec: Nice!
\noindent Hakai Institute $|$ Department of Zoology\\
\noindent University of British Columbia
\newpage
\vspace{-5ex}

\begingroup
\renewcommand{\section}[2]{}
\bibliographystyle{bibliography/bibstyles/besjournals}% 
\bibliography{bibliography/phenoBCBib.bib}
\endgroup
\newpage


\end{document}

%\begingroup
%\renewcommand{\section}[2]{}%
%\renewcommand{\chapter}[2]{}% for other classes
%\bibliography{mybibfile}
%\endgroup