\documentclass[11pt,a4paper]{article}
\usepackage[top=1.0in, bottom=1.0in, left=1.1in, right=1.1in]{geometry}
\usepackage{graphicx}
\usepackage[sort&compress]{natbib}
\usepackage[export]{adjustbox}
\usepackage[ngerman]{babel}
\usepackage[utf8]{inputenc}
\usepackage[T1]{fontenc}
\renewcommand{\bibname}{References}%names reference list 
% novelty statement, a cover letter, and a title page including an authorship statement, a data statement, the number of words in abstract/main text and the number of figure/table/boxes.
% recommended and opposed reviewers, and recommended and opposed editorial board members with whom they are in conflict. 
\begin{document}

\pagenumbering{gobble}
\noindent \includegraphics[width=0.4\textwidth, right]{letterhead/ubc-logo-2018-fullsig-blue-cmyk.png}
\noindent Dear Dr. Öpik
\vspace{1.5ex}\\
\noindent Please consider our paper, ``How temperature, photoperiod and evolutionary history shape forest leafout'' for publication as a full paper in \emph{New Phytologist}. 
\vspace{1.5ex}\\ 
Global shifts in the timings of species life history events with climate change has led to increasing interest in how communities assemble in time \citep{Cope2022, Cleland2024}. Yet progress has been slow, as species timings are highly variable, especially in systems where shifts have been greatest. Experiments can decompose this variability into predictable responses to environmental cues, such as temperature and daylength \citep{Basler2014,Vitasse2014,Zohner2016}. Given their logistical challenges, however, most experiments have focused on only a few species---providing limited insights into community dynamics under future climates.
\vspace{1.5ex}\\ 
\noindent \textit{What hypotheses or questions does this work address?} Accurately forecasting shifts in phenology and its impact on ecological communities requires answering: which environmental cues are most important at the population-, species-, functional-group and community-levels? We provide a continental-scale answer to this for North American forests using experiments that alter temperature and daylength, combined with a model that can robustly partition variation, including the role of evolutionary history. %59 words
\vspace{1.5ex}\\ 
\noindent \textit{How does this work advance our current understanding of plant science?}  By using a large-scale experiment of 47 species spanning 6\textdegree latitude and 55\textdegree longitude, our results provide compelling evidence for similar leafout across temperate forests and functional groups. A surprisingly large amount of variation in species budburst timings was unexplained by different responses to temperature and photoperiod, however, suggesting important unexplained variation in our mechanistic model of budburst.  %58 words
\vspace{1.5ex}\\
\noindent \textit{Why is this work important and timely?} Our results highlight fundamental gaps in our model of one of the best studied and most important phenological events---woody plant budburst---through unexplained species-level differences. Such differences may structure plant communities and ecosystems. Further, our findings of no detectable population-level variation and smaller than expected differences in functional groups has major implications for forecasting forest leafout. %57 words
% Alt: Our results highlight major gaps in our model of one of the best studied and most important phenological events---woody plant budburst. Further, our findings of high species-level variation in budburst, with no detectable population-level variation and smaller than expected differences in functional groups, may explain why current land-surface model often fail to forecast leafout variation. %57 words
\vspace{1.5ex}\\
\noindent Both authors contributed to this work and approve this version for submission. The manuscript is 4095 words with a 199 word summary, and 4 figures and is not under consideration elsewhere. We hope you find it suitable for publication in \emph{New Phytologist}, and look forward to hearing from you. 
%\noindent We recommend the following reviewers: Dr. Meredith Zettlemoyer, Dr. Mason Heberling, Dr. Jason Fridley, Dr. Rong Yu, and Dr. Ameila Caffara. 
\vspace{1.5ex}\\
\noindent Sincerely, \\
\includegraphics[scale=.4]{letterhead/shot.png} \\ 
\noindent Deirdre Loughnan\\
\noindent Sentinels of Change Postdoctoral Fellow\\ %emw19Dec: Nice!
\noindent Hakai Institute $|$ Department of Zoology\\
\noindent University of British Columbia
\newpage
\vspace{-5ex}

\begingroup
\renewcommand{\section}[2]{}
\bibliographystyle{bibliography/bibstyles/besjournals}% 
\bibliography{bibliography/phenoBCBib.bib}
\endgroup
\newpage


\end{document}

%\begingroup
%\renewcommand{\section}[2]{}%
%\renewcommand{\chapter}[2]{}% for other classes
%\bibliography{mybibfile}
%\endgroup