%%%%%%%%%%%%%%%%%%%%%%%%%%%%%%%%%%%%%%STARt PREAMBLE
\documentclass{letter}

%Required: You must have these
\usepackage{graphicx}
\usepackage{tabularx}
\usepackage{hyperref}
\usepackage{pdflscape}
\usepackage{array}
\usepackage{authblk}
\usepackage{amsmath}
%\usepackage[backend=bibtex]{biblatex}
%Strongly recommended
 %put your figures in one place
%\SweaveOpts{prefix.string=figures/, eps=FALSE} 
%you'll want these for pretty captioning
\usepackage[small]{caption}

\setkeys{Gin}{width=0.8\textwidth} %make the figs 50 perc textwidth
\setlength{\captionmargin}{30pt}
\setlength{\abovecaptionskip}{10pt}
\setlength{\belowcaptionskip}{10pt}
% manual for caption http://www.dd.chalmers.se/latex/Docs/PDF/caption.pdf

%Optional: I like to muck with my margins and spacing in ways that LaTeX frowns on
%Here's how to do that
\topmargin -1.5cm  
\oddsidemargin -0.04cm 
\evensidemargin -0.04cm % same as oddsidemargin but for left-hand pages
\textwidth 16.59cm
\textheight 21.94cm 
%\pagestyle{empty}  % Uncomment if don't want page numbers
\parskip 7.2pt   % sets spacing between paragraphs
%\renewcommand{\baselinestretch}{1.5} 	% Uncomment for 1.5 spacing between lines
\parindent 0pt% sets leading space for paragraphs
\usepackage{setspace}
%\doublespacing
\renewcommand{\baselinestretch}{1}
\usepackage{lineno}
 
%%%%%%%%%%%%%%%%%%%%%%%%%%%%%%%%%%%%%%END PREAMBLE 

%Start of the document
\begin{document}

\textbf{Question responses to GCB:}


\textbf{Running Head: 43/45 characters}\\

The environment and evolution shape forests 


\textbf{What scientific question is addressed in this manuscript? 247 or 234 characters}\\
%emw7Oct2024 -- I think we need to be more exciting! And remember we can more here than we can in the paper (e.g., population)... here's one idea. 
% What determines budburst timing across species, sites and diverse forest communities in North America? While many models critical for future forecasts assume temperature and photoperiod explain forest leafout, we test the role of populations, species and community-level variation, including the role of evolutionary history in shaping species differences. 
Climate change is shifting the timing of spring growth, making it critical we understand the drivers of variability in events like budburst. We tested how cues and evolutionary history relates to community-level budburst and across North America. \\

OR\\

We are one of the first to mechanistically test the relationships and variability in budburst cues and evolutionary relationships at the community-level—with 47 tree and shrub species—and between four forests that span North America. 


\textbf{What is/are the key findings that answer this question? 232}\\

While increasing cues led to an advance in budburst, we found little variation across sites, but strong phylogenetic structuring. This suggests additional traits are driving species-level differences, and not population differences.\\


\textbf{What are the novel results, ideas, or methods presented in work? 250}\\
%emw7Oct2024 -- I think the fact that our model of budburst is massively incomplete should be the answer to the PREVIOUS question (and should sound much more exciting). Also, stick with evolutionary history here and don't try to mix in phylogenetic as a term if you can. For this question you can talk a little more about the vast scale of the study and your analytical approach. 
Our Bayesian phylogenetic model estimated cues to explain only part of the variation in budburst. This suggests an incomplete understanding of budburst and the importance of other unidentified traits or cues to spring phenology in forest communities.\\

\textbf{Describe how your paper fits within the scope of GCB, what biological AND global change aspects does it address.}\\
%emw7Oct2024 -- I like the idea of pointing out here that results apply beyond forests and beyond plants, but this is ALSO your best chance to say why this matters -- carbon storage, forests etc. should be here as your opening finding. 
Our manuscript showcases an analytical approach that has broad applications across diverse species assemblages and phenologies, while providing novel insights into the complex ecological processes that shape species phenology and future responses. \\

\textbf{What are three most recently published papers that are relevant to this manuscript?}\\
\begin{enumerate} %emw7Oct2024 -- I would cite work NOT by the lab and ideally in Science, Nature, PNAS etc. You could cite the Zohner solstice work for example perhaps. 
\item Morales-Castilla et al. 2024. Phylogenetic estimates of species-level phenology improve ecological forecasting, NCC\\
\item Zeng \& Wolkovich. 2024. Weak evidence of provenance effects in spring phenology across Europe and North America. New Phytologist\\
\item Baumgarten et al. 2021. Chilled to be forced: the best dose to wake up buds from winter dormancy\\
\item Zhang et al. 2022. Deciphering the multiple effects of climate warming on the temporal shift of leaf unfolding NCC\\
\end{enumerate}

\textbf{Reviewers: 6}
%emw7Oct2024 -- Mason Heberling; Zettlemoyer ... I am not sure how much Europeans are into community differences and temporal niches so I might not suggest SO many of the below folks unless they are writing about this topic. 
\begin{itemize}
\item Amelia Caffara
\item Jason Fridley 
\item Rong Yu---University of Wisconsin 
\item Andrey Malyshev
\item Annette menzel
\end{itemize}




%%%%%%%%%%%%%%%%%%
%%%%%%%%%%%%%%%%%%%%%%
\end{document}
%%%%%%%%%%%%%%%%%%%%%%%%%%%%%%%%%%%%%%%%
% and for Nationally Determined Contributions, National Adaptation Plans, and other climate plans within the United Nations climate process are informed by the 'best available science.'
%Questions for the group:
%1) How detailed to get in Bayesian methods/theory in introduction?

%Text to add somewhere:
% in the past, climate science and biodiversity fields have been separated, but scientists and decision-makers  are increasingly highlighting  the role nature plays in global climate systems and the critical ecosystem services provided by nature to humans \citep{cohen2016nature,nesshover2017science,USGCRP2024}

